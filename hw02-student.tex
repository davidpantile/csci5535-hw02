\documentclass[11pt]{exam}
\usepackage[margin=1in]{geometry}
\usepackage{amsmath,amsthm,amssymb}
\usepackage{mathptmx}

\usepackage{xspace}
\usepackage{tabularx}
\usepackage{amsmath}
\usepackage{mathpartir} %% for inference rules

%% Listings
\usepackage{listings}
\lstdefinestyle{default}{%
  basicstyle=\ttfamily,%
  commentstyle=\sl,%
  keywordstyle=\bf,%
  columns=fullflexible,%
  keepspaces=true,%
  mathescape%
}
\lstset{style=default}
\newcommand{\ocaml}[1]{\lstset{language=[Objective]Caml}\lstinline~#1~}
\lstnewenvironment{OCaml}
  {\lstset{language=[Objective]Caml}}
  {}

\title{CSCI 5535: Homework Assignment 2: Language Design and Implementation}
\author{
    David Baines
    \thanks{https://courses.cs.cornell.edu/cs412/2004sp/lectures/lec13.pdf}
    \thanks{https://csci3155.cs.colorado.edu/csci3155-notes.pdf}
    \thanks{https://www.hedonisticlearning.com/posts/understanding-typing-judgments.html}
}

%% Number questions by section
\renewcommand{\thequestion}{\thesection.\arabic{question}}

%% toggle math mode and text mode for tabular and array
\newcolumntype{C}{>{$}c<{$}}
\newcolumntype{L}{>{$}l<{$}}
\newcolumntype{R}{>{$}r<{$}}

\newcommand{\fmtkw}[1]{\mathtt{#1}}

\newcommand{\Typ}{\ensuremath{\mathsf{Typ}}}
\newcommand{\typ}{\ensuremath{\mathit{\tau}}}
\newcommand{\numtyp}{\ensuremath{\fmtkw{num}}}
\newcommand{\booltyp}{\ensuremath{\fmtkw{bool}}}

\newcommand{\Expr}{\ensuremath{\mathsf{Exp}}}
\newcommand{\expr}{\ensuremath{\mathit{e}}}

\newcommand{\addra}[1]{\ensuremath{\fmtkw{addr}[#1]}}
\newcommand{\addr}{\ensuremath{\mathit{a}}}
\newcommand{\numa}[1]{\ensuremath{\fmtkw{num}[#1]}}
\newcommand{\num}{\ensuremath{\mathit{n}}}
\newcommand{\boola}[1]{\ensuremath{\fmtkw{bool}[#1]}}
\newcommand{\bool}{\ensuremath{\mathit{b}}}
\newcommand{\plusa}[2]{\ensuremath{\fmtkw{plus}(#1; #2)}}
\newcommand{\plusc}[2]{\ensuremath{#1 \mathbin{\fmtkw{+}} #2}}
\newcommand{\timesa}[2]{\ensuremath{\fmtkw{times}(#1; #2)}}
\newcommand{\timesc}[2]{\ensuremath{#1 \mathbin{\fmtkw{*}} #2}}
\newcommand{\eqa}[2]{\ensuremath{\fmtkw{eq}(#1; #2)}}
\newcommand{\eqc}[2]{\ensuremath{#1 \mathrel{\fmtkw{==}} #2}}
\newcommand{\lea}[2]{\ensuremath{\fmtkw{le}(#1; #2)}}
\newcommand{\lec}[2]{\ensuremath{#1 \mathrel{\fmtkw{<=}} #2}}
\newcommand{\nota}[1]{\ensuremath{\fmtkw{not}(#1)}}
\newcommand{\notc}[1]{\ensuremath{\mathord{\fmtkw{!}}#1}}
\newcommand{\anda}[2]{\ensuremath{\fmtkw{and}(#1; #2)}}
\newcommand{\andc}[2]{\ensuremath{#1 \mathbin{\fmtkw{\&\&}} #2}}
\newcommand{\ora}[2]{\ensuremath{\fmtkw{or}(#1; #2)}}
\newcommand{\orc}[2]{\ensuremath{#1 \mathbin{\fmtkw{||}} #2}}

\newcommand{\Cmd}{\ensuremath{\mathsf{Cmd}}}
\newcommand{\cmd}{\ensuremath{\mathit{c}}}
\newcommand{\skipa}{\ensuremath{\fmtkw{skip}}}
\newcommand{\seta}[2]{\ensuremath{\fmtkw{set}[#1](#2)}}
\newcommand{\setc}[2]{\ensuremath{#1 \mathrel{\fmtkw{:=}} #2}}
\newcommand{\seqa}[2]{\ensuremath{\fmtkw{seq}(#1; #2)}}
\newcommand{\seqc}[2]{\ensuremath{#1\fmtkw{;}\;#2}}
\newcommand{\ifa}[3]{\ensuremath{\fmtkw{if}(#1; #2; #3)}}
\newcommand{\ifc}[3]{\ensuremath{\fmtkw{if}\;#1\;\fmtkw{then}\;#2\;\fmtkw{else}\;#3}}
\newcommand{\whilea}[2]{\ensuremath{\fmtkw{while}(#1; #2)}}
\newcommand{\whilec}[2]{\ensuremath{\fmtkw{while}\;#1\;\fmtkw{do}\;#2}}

\newcommand{\Addr}{\ensuremath{\mathsf{Addr}}}

\newcommand{\store}{\ensuremath{\sigma}}
\newcommand{\storelet}[2]{\ensuremath{#1 \hookrightarrow #2}}
\newcommand{\xstore}[3]{#1, \storelet{#2}{#3}}

\newcommand{\IMP}{\textbf{\textsf{IMP}}\xspace}

\newcommand{\E}{\textbf{\textsf{E}}\xspace}
\newcommand{\T}{\textbf{\textsf{T}}\xspace}
\renewcommand{\P}{\textbf{\textsf{P}}\xspace}
\renewcommand{\S}{\textbf{\textsf{S}}\xspace}
\newcommand{\ET}{\textbf{\textsf{ET}}\xspace}
\newcommand{\ETP}{\textbf{\textsf{ETP}}\xspace}
\newcommand{\ETPS}{\textbf{\textsf{ETPS}}\xspace}

\newcommand{\state}[2]{\langle #1, #2 \rangle}

\newcommand{\hasType}[2]{\ensuremath{#1 : #2}}
\newcommand{\hypJ}[2]{\ensuremath{#1 \vdash #2}}
\newcommand{\isOk}[1]{\ensuremath{#1\;\mathsf{ok}}}
\newcommand{\eval}[2]{\ensuremath{#1 \Downarrow #2}}
\newcommand{\step}[2]{\ensuremath{#1 \longmapsto #2}}
\newcommand{\stepspap}[3][\typ]{\ensuremath{#2 \hookrightarrow_{:#1} #3}}
\newcommand{\isVal}[1]{\ensuremath{#1\;\mathsf{val}}}
\newcommand{\isFinal}[1]{\ensuremath{#1\;\mathsf{final}}}

\newcommand{\even}{\operatorname{even}}

%%%%%%%%%%%%%%
% Baines custom commands
%%%%%%%%%%%%%%

\newcommand{\mycomment}[1]{}

\pagestyle{plain}
\begin{document}
\maketitle

\section{Language Design: \IMP}

\begin{questions}
  \question The two judgment forms (one for expressions, the other for functions / commands, respectively \expr \; and \cmd \; in \IMP, are:

\mycomment{

    \begin{mathpar}
    \inferrule{
        \Gamma , \hasType{x_1}{\tau_1} \vdash \hasType{e_2}{\tau_2} \\
        \Gamma , \hasType{\cmd(\tau_1)}{\tau_2} \vdash \hasType{e}{\tau}
    }{
        \Gamma \vdash \texttt{fun}[\tau_1;\tau_2] \hasType{(x_1.e_2; f.e)}{\tau}
    }

    \inferrule{
        \Gamma \vdash \hasType{c(\tau_1)}{\tau_2} \\
        \Gamma \vdash \hasType{e}{\tau_1}
    }{
        \Gamma \vdash \texttt{call}[c] \hasType{(e)}{\tau_2}
    }
    \end{mathpar}

}

For $\expr$:

\begin{mathpar}
    \inferrule{
    }{
        \Gamma \vdash \hasType{\addra{a}}{\numtyp}
    }
    
    \inferrule{
    }{
        \Gamma \vdash \hasType{\numa{n}}{\numtyp}
    }
    
    \inferrule{
    }{
        \Gamma \vdash \hasType{\boola{b}}{\booltyp}
    }

    \inferrule{
        \Gamma \vdash \hasType{e_1}{\numtyp} \\ 
        \Gamma \vdash \hasType{e_2}{\numtyp}
    }{
        \Gamma \vdash \hasType{\plusa{e_1}{e_2}}{\numtyp}
    }

    \inferrule{
        \Gamma \vdash \hasType{e_1}{\numtyp} \\ 
        \Gamma \vdash \hasType{e_2}{\numtyp}
    }{
        \Gamma \vdash \hasType{\timesa{e_1}{e_2}}{\numtyp}
    }

    \inferrule{
        \Gamma \vdash \hasType{e_1}{\numtyp} \\ 
        \Gamma \vdash \hasType{e_2}{\numtyp}
    }{
        \Gamma \vdash \hasType{\eqa{e_1}{e_2}}{\booltyp}
    }

    \inferrule{
        \Gamma \vdash \hasType{e_1}{\booltyp} \\ 
        \Gamma \vdash \hasType{e_2}{\booltyp}
    }{
        \Gamma \vdash \hasType{\eqa{e_1}{e_2}}{\booltyp}
    }

    \inferrule{
        \Gamma \vdash \hasType{e_1}{\numtyp} \\ 
        \Gamma \vdash \hasType{e_2}{\numtyp}
    }{
        \Gamma \vdash \hasType{\lea{e_1}{e_2}}{\booltyp}
    }

    \inferrule{
        \Gamma \vdash \hasType{e}{\numtyp}
    }{
        \Gamma \vdash \hasType{\nota{e}}{\booltyp}
    }

    \inferrule{
        \Gamma \vdash \hasType{e}{\booltyp}
    }{
        \Gamma \vdash \hasType{\nota{e}}{\booltyp}
    }

    \inferrule{
        \Gamma \vdash \hasType{e_1}{\booltyp} \\
        \Gamma \vdash \hasType{e_2}{\booltyp}
    }{
        \Gamma \vdash \hasType{\anda{e_1}{e_2}}{\booltyp}
    }

    \inferrule{
        \Gamma \vdash \hasType{e_1}{\booltyp} \\
        \Gamma \vdash \hasType{e_2}{\booltyp}
    }{
        \Gamma \vdash \hasType{\ora{e_1}{e_2}}{\booltyp}
    }
  
\end{mathpar}

    For $\cmd$:

\begin{mathpar}
    \inferrule{
        \Gamma \vdash \hasType{a}{\numtyp} \\ 
        \Gamma \vdash \hasType{e}{\typ}
    }{
        \Gamma \vdash \hasType{\seta{a}{e}}{\numtyp}
    }

    \inferrule{
        \Gamma \vdash \hasType{\cmd_1}{\typ} \\ 
        \Gamma \vdash \hasType{\cmd_2}{\typ}
    }{
        \Gamma \vdash \hasType{\seqa{\cmd_1}{\cmd_2}}{\typ}
    }

    \inferrule{
        \Gamma \vdash \hasType{\expr}{\booltyp} \\
        \Gamma \vdash \hasType{\cmd_1}{\typ} \\ 
        \Gamma \vdash \hasType{\cmd_2}{\typ}
    }{
        \Gamma \vdash \hasType{\ifa{\expr}{\cmd_1}{\cmd_2}}{\typ}
    }

    \inferrule{
        \Gamma \vdash \hasType{\expr}{\booltyp} \\
        \Gamma \vdash \hasType{\cmd}{\typ} \\ 
    }{
        \Gamma \vdash \hasType{\whilea{\expr}{\cmd}}{\typ}
    }
\end{mathpar}

    \question
    \begin{itemize}
        \item[(a)] Define values and commands. \\
        
        For ``$\expr \isVal{}$'':
        
        \begin{mathpar}
        

            \inferrule{ }{\addra{a} \isVal{}}

            \inferrule{ }{\numa{n} \isVal{}}
            
            \inferrule{ }{\boola{b} \isVal{}}
            
            \\\\
            
        \end{mathpar}
            
        ``plus()'':    

	\begin{mathpar} 

            \inferrule{
            	n_1 + n_2 = n
            }{ 
            	\step{
            		\plusa{\numa{n_1}}{\numa{n_2}}
            	}{
            		\numa{n}
            	}
            }
            
            \inferrule{
            	\step{
            		e_1
            	}{
            		e_1^\prime
            	}
            }
            {
            	\step{
            		\plusa{e_1}{e_2}
            	}{
            		\plusa{e_1^\prime}{e_2}
            	}
            }

            \inferrule{
            	\expr \isVal{} \\
            	\step{
            		e_2
            	}{
            		e_2^\prime
            	}
            }
            {
            	\step{
            		\plusa{e_1}{e_2}
            	}{
            		\plusa{e_1}{e_2^\prime}
            	}
            }
            
        \end{mathpar}
            
        ``times()'':     

	\begin{mathpar}
           
            \inferrule{
            	n_1 * n_2 = n
            }{ 
            	\step{
            		\timesa{
            			\numa{n_1}
            		}{
            			\numa{n_2}
            		}
            	}{
            		\numa{n}
            	}
            }
            
            \inferrule{
            	\step{
            		e_1
            	}{
            		e_1^\prime
            	}
            }
            {
            	\step{
            		\timesa{
            			e_1
            		}{
            			e_2
            		}
            	}{
            		\timesa{e_1^\prime}{e_2}
            	}
            }

            \inferrule{
            	\expr \isVal{} \\
            	\step{
            		e_2
            	}{
            		e_2^\prime
            	}
            }
            {
            	\step{
            		\timesa{e_1}{e_2}
            	}{
            		\timesa{e_1}{e_2^\prime}
            	}
            }
            
        \end{mathpar}
            
        ``eq()'':   

	\begin{mathpar}  
        
            \inferrule{
            	n_1 == n_2 \vdash b
            }{ 
            	\step{
            		\eqa{\numa{n_1}}{\numa{n_2}}
            	}{
            		\boola{b}
            	}
            }
            
            \inferrule{
            	\step{
            		e_1
            	}{
            		e_1^\prime
            	}
            }
            {
            	\step{
            		\eqa{e_1}{e_2}
            	}{
            		\eqa{e_1^\prime}{e_2}
            	}
            }

            \inferrule{
            	\expr \isVal{} \\
            	\step{
            		e_2
            	}{
            		e_2^\prime
            	}
            }
            {
            	\step{
            		\eqa{e_1}{e_2}
            	}{
            		\eqa{e_1}{e_2^\prime}
            	}
            }
            
        \end{mathpar}
            
        ``le()'':  

	\begin{mathpar}   
      
            \inferrule{
            	n_1 <= n_2 \vdash b
            }{ 
            	\step{
            		\lea{\numa{n_1}}{\numa{n_2}}
            	}{
            		\boola{b}
            	}
            }
            
            \inferrule{
            	\step{
            		e_1
            	}{
            		e_1^\prime
            	}
            }
            {
            	\step{
            		\lea{e_1}{e_2}
            	}{
            		\lea{e_1^\prime}{e_2}
            	}
            }

            \inferrule{
            	\expr \isVal{} \\
            	\step{
            		e_2
            	}{
            		e_2^\prime
            	}
            }
            {
            	\step{
            		\lea{e_1}{e_2}
            	}{
            		\lea{e_1}{e_2^\prime}
            	}
            }
            
        \end{mathpar}
            
        ``not()'':    

        \begin{mathpar} 
   
            \inferrule{
            	n_1 ! \; n_2 \; || \;  b_1 ! \; b_2 \vdash b
            }{ 
            	\step{
            		\nota{\numa{n_1}}{\numa{n_2}} \; || \;
            		\nota{\boola{b_1}}{\boola{b_2}}
            	}{
            		\boola{b}
            	}
            }
            
            \inferrule{
            	\step{
            		e_1
            	}{
            		e_1^\prime
            	}
            }
            {
            	\step{
            		\nota{e_1}{e_2}
            	}{
            		\nota{e_1^\prime}{e_2}
            	}
            }

            \inferrule{
            	\expr \isVal{} \\
            	\step{
            		e_2
            	}{
            		e_2^\prime
            	}
            }
            {
            	\step{
            		\nota{e_1}{e_2}
            	}{
            		\nota{e_1}{e_2^\prime}
            	}
            }
            
       \end{mathpar}
            
        ``and()'':

	\begin{mathpar}
         
            \inferrule{
            	b_1 \; \land \; b_2 \vdash b
            }{ 
            	\step{
            		\anda{\boola{b_1}}{\boola{b_2}}
            	}{
            		\boola{b}
            	}
            }
            
            \inferrule{
            	\step{
            		e_1
            	}{
            		e_1^\prime
            	}
            }
            {
            	\step{
            		\anda{e_1}{e_2}
            	}{
            		\anda{e_1^\prime}{e_2}
            	}
            }

            \inferrule{
            	\expr \isVal{} \\
            	\step{
            		e_2
            	}{
            		e_2^\prime
            	}
            }
            {
            	\step{
            		\anda{e_1}{e_2}
            	}{
            		\anda{e_1}{e_2^\prime}
            	}
            }
        \end{mathpar}

        ``or()'':

	\begin{mathpar}
         
            \inferrule{
            	b_1 \; \lor \; b_2 \vdash b
            }{ 
            	\step{
            		\ora{\boola{b_1}}{\boola{b_2}}
            	}{
            		\boola{b}
            	}
            }
            
            \inferrule{
            	\step{
            		e_1
            	}{
            		e_1^\prime
            	}
            }
            {
            	\step{
            		\ora{e_1}{e_2}
            	}{
            		\ora{e_1^\prime}{e_2}
            	}
            }

            \inferrule{
            	\expr \isVal{} \\
            	\step{
            		e_2
            	}{
            		e_2^\prime
            	}
            }
            {
            	\step{
            		\ora{e_1}{e_2}
            	}{
            		\ora{e_1}{e_2^\prime}
            	}
            }
        \end{mathpar}

        For ``$\cmd$ \isFinal{}'': \\

        ``set()'': 

        \begin{mathpar}
           \inferrule{
                \addra{a} = \numa{n}
            }{
                \step{\seta{\addr}{\expr}}{\expr}
            }
        
            \inferrule{
                \addra{a} \isVal{} \\
                \step{\expr}{\expr^\prime}
            }{
                \step{\seta{\addr}{\expr}}{\expr^\prime}
            }
        \end{mathpar}

        ``skip()'': 

        \begin{mathpar}
           \inferrule{\addra{a} = \numa{n}}{
                \step{\skipa}{\addr}
            }
        \end{mathpar}

        ``seq()'': 

        \begin{mathpar}
            \inferrule{
                \step{\cmd}{\texttt{ok}}
           }{
                \step{\seqa{\cmd_1}{\cmd_2}}{\cmd_2}
            }

            \inferrule{
                \step{\cmd_1}{\cmd_1^\prime}
           }{
                \step{\seqa{\cmd_1}{\cmd_2}}{\seqa{\cmd_1^\prime}{\cmd_2}}
            }

            \inferrule{
                \step{\cmd_2}{\cmd_2^\prime}
           }{
                \step{\seqa{\cmd_1}{\cmd_2}}{\seqa{\cmd_1}{\cmd_2^\prime}}
            }
        \end{mathpar}

        ``if()'': 

        \begin{mathpar}
            \inferrule{
                \texttt{if} \; \expr \; \texttt{then} \; \cmd_1 \; \texttt{else} \; \cmd_2
           }{
                \step{\ifa{\expr}{\cmd_1}{\cmd_2}}{\cmd_1 \lor \cmd_2}
            }

           \inferrule{
                \expr \isVal{} \\  
                \step{\expr}{\expr^\prime}
           }{
                \step{\ifa{\expr}{\cmd_1}{\cmd_2}}{\ifa{\expr^\prime}{\cmd_1}{\cmd_2}}
            }

           \inferrule{
                \expr \isVal{} \\  
                \step{\cmd_1}{\cmd_1^\prime}
           }{
                \step{\ifa{\expr}{\cmd_1}{\cmd_2}}{\ifa{\expr}{\cmd_1^\prime}{\cmd_2}}
            }

           \inferrule{
                \expr \isVal{} \\  
                \step{\cmd_2}{\cmd_2^\prime}
           }{
                \step{\ifa{\expr}{\cmd_1}{\cmd_2}}{\ifa{\expr}{\cmd_1}{\cmd_2^\prime}}
            }
        \end{mathpar}

        ``while()'': 

        \begin{mathpar}
            \inferrule{
                \texttt{while} \; \expr \; \texttt{do} \; \cmd
           }{
                \step{\whilea{\expr}{\cmd}}{\cmd}
            }

           \inferrule{
                \expr = \boola{\bool} \\
                \expr \isVal{} \\
           }{
                \step{\whilea{\expr}{\cmd}}{\whilea{\expr}{\cmd}}
            }

           \inferrule{
                \step{\expr}{\expr^\prime}
           }{
                \step{\whilea{\expr}{\cmd}}{\whilea{\expr^\prime}{\cmd}}
            }

           \inferrule{
                \step{\cmd}{\cmd^\prime}
           }{
                \step{\whilea{\expr}{\cmd}}{\whilea{\expr}{\cmd^\prime}}
            }
            
        \end{mathpar}
        
        \item[(b)] Define the small-step semantics. \\
        
        For ``$\expr$'':
        
        \begin{mathpar}

            \inferrule{ }{\step{\state{\addr}{\store}}{\store(\addr)}}
            
            \inferrule{ }{\step{\state{\num}{\store}}{\num}}

            \inferrule{ }{\step{\state{\texttt{true}}{\store}}{\texttt{true}}}

            \inferrule{ }{\step{\state{\texttt{false}}{\store}}{\texttt{false}}}

            % plus
            \inferrule{
                \step{
                    \state{\addr_0}{\store}
                }{
                    \num_0
                } \\
                \step{
                    \state{\addr_1}{\store}
                }{
                    \num_1
                }
            }{
                \step{
                    \state{\num_0 - \num_1}{\store}
                }{
                    \num
                }
            }

            % times
            \inferrule{
                \step{
                    \state{\addr_0}{\store}
                }{
                    \num_0
                } \\
                \step{
                    \state{\addr_1}{\store}
                }{
                    \num_1
                }
            }{
                \step{
                    \state{\num_0 \times \num_1}{\store}
                }{
                    \num
                }
            }

            % equal
            \inferrule{
                \step{
                    \state{\addr_0}{\store}
                }{
                    n
                } \\
                \step{
                    \state{\addr_1}{\store}
                }{
                    m
                }
            }{
                \step{
                    \state{n = m}{\store}
                }{
                    \texttt{true}
                }
            }

            \inferrule{
                \step{
                    \state{\addr_0}{\store}
                }{
                    n
                } \\
                \step{
                    \state{\addr_1}{\store}
                }{
                    m
                }
            }{
                \step{
                    \state{n = m}{\store}
                }{
                    \texttt{false}
                }
            }

            % less than
            \inferrule{
                \step{
                    \state{\addr_0}{\store}
                }{
                    n
                } \\
                \step{
                    \state{\addr_1}{\store}
                }{
                    m
                }
            }{
                \step{
                    \state{n \le m}{\store}
                }{
                    \texttt{true}
                }
            }

            \inferrule{
                \step{
                    \state{\addr_0}{\store}
                }{
                    n
                } \\
                \step{
                    \state{\addr_1}{\store}
                }{
                    m
                }
            }{
                \step{
                    \state{n \le m}{\store}
                }{
                    \texttt{false}
                }
            }

            % and
            \inferrule{
                \step{
                    \state{\addr_0}{\store}
                }{
                    \texttt{true}
                } \\
                \step{
                    \state{\addr_1}{\store}
                }{
                    \texttt{true}
                }
            }{
                \step{
                    \state{n \land m}{\store}
                }{
                    \texttt{true}
                }
            }

            \inferrule{
                \step{
                    \state{\addr_0}{\store}
                }{
                    \texttt{false}
                } \\
                \step{
                    \state{\addr_1}{\store}
                }{
                    \texttt{true}
                }
            }{
                \step{
                    \state{n \land m}{\store}
                }{
                    \texttt{false}
                }
            }

            \inferrule{
                \step{
                    \state{\addr_0}{\store}
                }{
                    \texttt{true}
                } \\
                \step{
                    \state{\addr_1}{\store}
                }{
                    \texttt{false}
                }
            }{
                \step{
                    \state{n \land m}{\store}
                }{
                    \texttt{false}
                }
            }

            % or
            \inferrule{
                \step{
                    \state{\addr_0}{\store}
                }{
                    \texttt{true}
                } \\
                \step{
                    \state{\addr_1}{\store}
                }{
                    \texttt{true}
                }
            }{
                \step{
                    \state{n \lor m}{\store}
                }{
                    \texttt{true}
                }
            }

            \inferrule{
                \step{
                    \state{\addr_0}{\store}
                }{
                    \texttt{true}
                } \\
                \step{
                    \state{\addr_1}{\store}                }{
                    \texttt{false}
                }
            }{
                \step{
                    \state{n \lor m}{\store}
                }{
                    \texttt{true}
                }
            }

            \inferrule{
                \step{
                    \state{\addr_0}{\store}
                }{
                    \texttt{false}
                } \\
                \step{
                    \state{\addr_1}{\store}
                }{
                    \texttt{true}
                }
            }{
                \step{
                    \state{n \lor m}{\store}
                }{
                    \texttt{true}
                }
            }

            \inferrule{
                \step{
                    \state{\addr_0}{\store}
                }{
                    \texttt{false}
                } \\
                \step{
                    \state{\addr_1}{\store}
                }{
                    \texttt{false}
                }
            }{
                \step{
                    \state{n \lor m}{\store}
                }{
                    \texttt{false}
                }
            }

        \end{mathpar}
        For ``$\cmd$'':
            
        \begin{mathpar}

            % set
            \inferrule{
                \step{\state{\addr}{\store}}{\store(\addr)} \\
                \step{\state{\expr}{\store}}{\expr}
            }{
                \step{
                    \state{\addr}{\store}
                }{
                    \state{\expr}{\store}
                }
            }

            % skip
            \inferrule{
            }{
                \step{
                    \state{\skipa}{\store}
                }{
                    \store
                }
            }  

            % seq
            \inferrule{
                \step{\state{\cmd_0}{\store}}{\store^{\prime \prime}} \\
                \step{\state{\cmd_1}{\store^{\prime \prime}}}{\store^{\prime}}
            }{
                \step{\state{\cmd_0 \; \text{;} \; \cmd_1}{\store}}{\store^\prime}
            }

            % if
            \inferrule{
                \step{\state{\bool}{\store}}{\texttt{true}} \\
                \step{\state{\cmd_0}{\store}}{\store^{\prime}}
            }{
                \step{\state{\textbf{if} \; \bool \; \textbf{then} \; \cmd_0 \; \textbf{else} \; \cmd_1}{\store}}{\store^\prime}
            }

            \inferrule{
                \step{\state{\bool}{\store}}{\texttt{false}} \\
                \step{\state{\cmd_0}{\store}}{\store^{\prime}}
            }{
                \step{\state{\textbf{if} \; \bool \; \textbf{then} \; \cmd_0 \; \textbf{else} \; \cmd_1}{\store}}{\store^\prime}
            }

            % while
            \inferrule{
                \step{\state{\bool}{\store}}{\texttt{false}}
            }{
                \step{\state{\textbf{while} \; \bool \; \textbf{do} \; \cmd}{\store}}{\store}
            }

            \inferrule{
                \step{\state{\bool}{\store}}{\texttt{true}} \\
                \step{\state{\cmd}{\store}}{\store^{\prime \prime}} \\
                \step{\state{\textbf{while} \; \bool \; \textbf{do} \; \cmd}{\store^{\prime \prime}}}{\store^\prime}
            }{
                \step{
                    \state{\textbf{while} \; \bool \; \textbf{do} \; \cmd}{\store}
                }{
                    \store^\prime
                }
            }
            \end{mathpar}
            \item[(c)]
            \begin{itemize}
                \item[i.] Canonical Form Lemmas \\

                If $\expr \isVal{}$ and $\hasType{\expr}{\tau}$, then \\

                \begin{enumerate}
                    \item If $\tau = \numtyp$, then $\expr = \numa{n}$ for some number $\num$.
                    \item If $\tau = \booltyp$, then $\expr = \boola{b}$ for some boolean $\bool$.
                \end{enumerate} 

                This is basically saying that our expressions and commands are well-typed, meaning, for 1 and 2, if our expression evaluates and the type is $\numtyp$ or $\booltyp$ (respectively), then the evaluated expression $\expr$'s output is also of that type.
                \\
                \item[ii.] Progress and Preservation. \\

                \textit{Progress}: $\forall \expr \in \Expr,$ either $\expr \isVal{}$ or $\exists \; \expr^\prime \; \text{:} \; \step{\expr}{\expr^\prime}$. \\
                
                Progress says that our expressions are well-typed, and that we can transition from expression to expression knowing that the types of one expression are compatible with the others they may encounter. Progress enables a program to not ``get stuck.''
                \\
                
                \textit{Preservation}: If $\hasType{\expr}{\typ}$ and $\step{\expr}{\expr^\prime}$, then $\hasType{\expr^\prime}{\typ}$. \\

                Preservation is the idea that our expressions are well-typed consistently. This allows us to use induction on our proofs because, if one instance of an expression has a type, than all other instances of that expression must have the same type. \\

                \item [iii.] Prove progress and preservation for commands. \\

                \textit{Progress}: $\forall \cmd \in \Cmd,$ either $\cmd \isFinal{}$ or $\exists \; \cmd^\prime \; \text{:} \; \step{\cmd}{\cmd^\prime}$. \\

                Progress is proved by the statics in 1.1. For example, if $\hasType{\cmd_1}{\typ}$ and $\hasType{\cmd_2}{\typ}$, then $\hasType{\seqa{\cmd_1}{\cmd_2}}{\typ}$. Since all of our commands are well-typed, progress is achieved. \\

                \textit{Preservation}: If $\hasType{\cmd}{\typ}$ and $\step{\cmd}{\cmd^\prime}$, then $\hasType{\cmd^\prime}{\typ}$. \\

                Preservation is proven with our dynamics. For example, since $\step{\cmd}{\texttt{ok}}$ and $\step{\seqa{\cmd_1}{\cmd_2}}{\cmd_2}$, know that $\step{\seqa{\cmd_1}{\cmd_2}}{\seqa{\cmd_1^\prime}{\cmd_2}}$ for all $\cmd \in \Cmd$.
            \end{itemize}
        \item[(d)] Did not have time.
        \item[(e)] Did not have time.
    \end{itemize}
\end{questions}

\section{Language Implementation: \ETPS}

See \texttt{hw02.ml} and \texttt{test\_hw02.ml}. \\

I was able to download and start learning OCaml via ``https://v2.ocaml.org/learn/tutorials/a\_first\_hour\_with\_ocaml.html'', but I did not have time to even start the unit tests even though I understand the concepts. It's the grammar of proving them in OCaml that took an undue amount of time.

\section{Final Project Preparation: Pre-Proposal}

\begin{questions}
    \question I'm working with Matt Buchholz and we're thinking about how PL verification can be applied to LLMs to ensure their outputs are well-formed. Below is the synopsis. \\

    \textbf{Basic problem:} Can a language model trained to perform some task with code (like e.g. GitHub Copilot's code 'auto-complete' feature) be improved with constraints on the well-formedness of their outputs.
    
    We draw inspiration from related work in natural languages, where e.g. Tziafas et al.  \cite{Tziafas} show that pre-training a BERT-based model on a part-of-speech tagging-like task improves downstream performance on other tasks. We could apply a similar approach, but instead of using the grammatical parse of natural language, we could train the model on a similar task with the syntactic parse of code. Or, we could follow an approach similar to Han et al. \cite{han-etal-2019-joint}, by using ILP formulations about the syntactic well-formedness of a program to constrain some of the model's optimization.
    Liu et al. \cite{liu2023code} do something similar, formulating 'code execution' as a pre-training task, and demonstrating a model pre-trained on that task performs better on downstream tasks, such as natural language-to-code translation.
    
    We're still in the exploratory phase of the project; it's unclear how feasible this project is, since we're not familiar with the availability of open-source models for working with code or datasets for any training or benchmarking tasks. (And we're not going to try to train something from scratch.) Though, Wang et al. \cite{wang2023codet5} seems to point to (at least some) models and datasets being readily available, including the CodeT5+ model they introduce.
    
    
    For context: we \emph{did} try to scan the papers of top PL conferences, but found papers which were largely not interesting to us (or beyond our understanding of PL such that it's hard to gauge the difficulty or novelty of a paper's achievement). Since we both have backgrounds in AI, something at the intersection of PL and AI seems most appropriate.

    \bibliography{refs}
    \bibliographystyle{IEEEtran}
\end{questions}

%% Appendix
\clearpage
\appendix

\section{Syntax of \IMP}

\[\begin{array}{lrcllL}
\Typ & \typ & ::= & \numtyp & \numtyp & numbers
\\
&&& \booltyp & \booltyp & booleans
\\
\Expr & \expr & ::= & \addra{\addr} & \addr & addresses (or ``assignables'') 
\\ 
&&& \numa{\num} & \num & numeral
\\
&&& \boola{\bool} & \bool & boolean
\\
&&& \plusa{\expr_1}{\expr_2} & \plusc{\expr_1}{\expr_2} & addition
\\
&&& \timesa{\expr_1}{\expr_2} & \timesc{\expr_1}{\expr_2} & multiplication
\\
&&& \eqa{\expr_1}{\expr_2} & \eqc{\expr_1}{\expr_2} & equal
\\
&&& \lea{\expr_1}{\expr_2} & \lec{\expr_1}{\expr_2} & less-than-or-equal
\\
&&& \nota{\expr_1} & \notc{\expr_1} & negation
\\
&&& \anda{\expr_1}{\expr_2} & \andc{\expr_1}{\expr_2} & conjunction
\\
&&& \ora{\expr_1}{\expr_2} & \orc{\expr_1}{\expr_2} & disjunction
\\
\Cmd & \cmd & ::= & \seta{\addr}{\expr} & \setc{\addr}{\expr} & assignment
\\
&&& \skipa & \skipa & skip
\\
&&& \seqa{\cmd_1}{\cmd_2} & \seqc{\cmd_1}{\cmd_2} & sequencing
\\
&&& \ifa{\expr}{\cmd_1}{\cmd_2} & \ifc{\expr}{\cmd_1}{\cmd_2} & conditional
\\
&&& \whilea{\expr}{\cmd_1} & \whilec{\expr}{\cmd_1} & looping
\\
\Addr & \addr
\end{array}\]

\end{document}\\